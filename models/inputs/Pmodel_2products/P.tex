\documentclass[12pt]{article}
\title{Агентская задача домохозяйства}
\date{19.04.2022}
\ProcessOptions
\usepackage[english,russian]{babel}
\begin{document}

Задача потребителя:
    $J = \int_0^T (\frac{1}{1-\beta_p} * (\frac{\pi_p(t)}{p(t)})^{1-\beta_p} * \exp{-\delta_p * t})dt -> max$: полезность
    $J$: CRRA от CES-свертки двух продуктов

Активы потребителя:
    $K(t, 1)$: капитал продукта 1
    $K(t, 2)$: капитал продукта 2
    $J(t, 1, 1)$: a
    $J(t, 1, 2)$: b
    $J(t, 2, 1)$: c
    $J(t, 2, 2)$: d
    $N(t)$: остаток денежных средств
    $\pi_{P}(t)$: выбор траектории дивидентов
    $\frac{\partial K(t,1)}{\partial t} = (a_{1} * J(t, 1, 1)^{\rho_{1}}+(1-a_{1}) * J(t,2,1)^{\rho_{1}})^{1 / \rho_{1}}$: траектория капитала 1
    $\frac{\partial K(t,2)}{\partial t} = (a_{2} * J(t, 1, 2)^{\rho_{2}}+(1-a_{2}) * J(t,2,2)^{\rho_{2}})^{1 / \rho_{2}}$: траектория капитала 2
    $\frac{dN(t)}{dt} = - \pi_{P}(t) + p_{1}(t)*K(t,1)/b_{1} + p_{2}(t)*K(t,2)/b_{2} - p_{1}(t)(J(t,1,1) + J(t,1,2)) - p_{2}(t)(J(t,2,1) + J(t,2,2)$: уравнение финансового баланса
    $N(T) \geq 0$: неотрицательность остатков
    $K(0,1) > 0$ : начальное условие 1
    $K(0,2) > 0$ : начальное условие 2
    $N(T) + p_{1}(T)*K(T, 1) + p_{2}(T)*K(T,2) \geq 0 $: терминальное условие

Информационные переменные:
    $p_{1}(t)$: цена на продукт1
    $p_{2}(t)$: цена на продукт2
    $\delta_p$: дисконт
    $\rho_h$: пп

Параметры системы:
    $T$: горизонт планирования
    $t$: Время
    $\beta_p$: beta


\end{document}